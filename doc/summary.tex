\documentclass[a4paper, 12pt]{article}
\begin{document}


\section{Code types}
Basically the erasure codes fall into two types:
\begin{enumerate}
\item
  MDS(Maximum Distance Separable) code
\item
  non-MDS code
\end{enumerate}
MDS code has typically been used in RAID-5 and RAID-6 system, where
any 1/2 disk failures are recoverable in N+1/2 disks. In this case,
the XOR-based family code is mainly applied. However, Reed-Solomon\cite{reed60}
code is more general type of erasure code. In fact, both of them are
based on Galois Field computation. While the RS code is not restrict
in certain GF, XOR-based code is actually computed in GF(2), which
means its code word length is $1$. To achieve the effect of recovering
original $n$ code words from $n + m$ coded words, erasure code has to
construct $(m + n) \times n$ matrix for encoding. And also this matrix
must be inversible. The first kind of constructing matrix is using
Vandermonde matrix. For $GF(2^m)$, it could construct useful matrix
for at most $2^m - 1$ code words. Another way is to replace the first
$n \times n$ matrix as identity matrix. In addition to relieve
computational cost, it could help XOR-based code construct inversible
matrix rather than Vandermonde matrix. 

On the other hand, erasure code can sacrifice some storage efficiency
to reach same reliability as MDS code. Low-Density Parity-Check(LDPC)
code was mentioned in \cite{luby97, gallager63}. Network system can
usuallly take advantage of this feature\cite{byers98}. And recently
some efforts were made on putting this code method to small scale,
e.g. local file system\cite{woitaszek07, hafner05}. 

\section{Performance}
The most significant performance of erasure code is definitely
encoding/decoding throughput. Galois Field multipication and addition
are major factor of impacting coding process, where obviously the
multiplication operations dominates the additions. The most simple way
improving performance it to cancel the muliplication. Limiting the
code word in $1$ bit leads to the XOR-based family codes. But the
XOR-based codes have only constant parities. Intially 1 parity in
RAID-5\cite{chen94}, then 2 in \cite{menon95}, at most 3 in
\cite{huang05} if we wish to keep MDS characteristic. Applying Cauchy
matrix\cite{blomer07} can convert all multiplications into additions,
and this can be applied in classic RS code. A second way is to keep a
multiplication table in memory. Then multiplication operation would be
simplified to a step of memory access. But we can only apply this
technique in code word smaller than $16$. In the case of code word
$16$, the full table would consume $4$GB memory. And even by
manipulating data structure we can shrink the table size, it is still
not realistic in larger code word than $16$. The third way relies on
the hardware implementation to improving Galois Field
multiplications\cite{plank13}.  

The reason why erasure code can surpass replication is excellent
storage efficiency. MDS undoubtly reach the optimal storage
effeciency. But sometime, we don't need the highest efficiency in
practical. If there is some other performance more considerable, the
storage effeciency can be sacrificed. Typically the LDPC code
make a tradeoff between storage efficiency and reliability. In Windows
Azure, people prefer the less bandwidth of reconstructing parities to
storage efficiency\cite{huang12}. 

\section{Application}
Although erasure code origninates from error correcting code in
network transmission, it is widely used in the current scenarioes of
distributed storage system for backup,e.g. OceanStore \cite{bindel00}.
Emerging cloud services,such as Window Azure and Google File System,
also incline to utilize erasure code for reliability\cite{huang12,ghemawat03}. 
In cloud environment of large scale storage, data chunks are usually
considered as appending only format. Compared to update effeciency
toward coded data, the service providers prefer focusing on rapidly
regenerating(e.g., reducing I/O in recovery process) coded blocks
\cite{kbp12, huang12} or improving storage effeciency by taking
advantage of characteristics of cloud\cite{pbh13}. 

RAID system is another field that need erasure code for storage
reliability. Considering highly frequent write/read operations on
disks, RAID system relatively intends to fast updates performance in
coded data. For this reason, XOR-based erasure code family is the most
appropriate option. While XOR-based code family may improve encode and
decode performance, RAID have to keep at most 3 parities to provide
optimal storage efficiency.



\bibliographystyle{plain}
\bibliography{ref}

\end{document}
